\chapter{Introduction}\label{introduction}

This thesis presents an execution stack for neural networks using the
\emph{Kubernetes}\footnote{https://kubernetes.io} container
orchestration and a Java based microservice architecture, which is
exposed to users and other systems via RESTful web services and a web
frontend. The whole workflow including importing, training and
evaluating a neural network model, becomes possible by using this
service oriented approach (SOA). The presented stack runs on popular
cloud platforms, like \emph{Google Cloud Platform}\footnote{https://cloud.google.com/kubernetes-engine},
\emph{Amazon AWS}\footnote{https://aws.amazon.com/eks} and
\emph{Microsoft Azure}\footnote{https://azure.microsoft.com/services/container-service}.
Furthermore it is scalable and each component is extensible and
interchangeable. This work is influenced by N2Sky \cite{schikuta_2013},
a framework to exchange neural network specific knowledge and aims to
support \emph{ViNNSL}, the Vienna Neural Network Specification Language
\cite{kopica_2015} \cite{beran_2008}.

\paragraph{Objectives:}\label{objectives}

The first objective is to specify functional and non-functional
requirements for the neural network system. This is followed by the
characterisation of the API and the implemention of microservices that
later define the neural network composition as a collection of loosly
coupled services.

The next step is to setup a \emph{Kubernetes} cluster to create the
foundation of container orchestration.

Finally the microservices are deployed to containers and combined in a
cluster.

\paragraph{Non-Objectives:}\label{non-objectives}

The prototype does not fully implement the \emph{ViNNSL 2.0}, as
described in \cite{kopica_2015} and provides limited data in-/output.
This is described in section TODO.

\section{Motivation}\label{motivation}

Machine learning has become a highly discussed topic in information
technology in the past years and the trend is further increasing. It has
become an essential part of everyday life when using search engines or
speech recognition systems, like personal assistants. Self-learning
algorithms in applications learn from the input of their users and
decide which news an individual should read next, which song to listen
to or which social media post should appear first. Messages are being
analyzed and possible answers automatically predicted.

A recent Californian study shows that 6.5 million developers worldwide
are currently involved in projects that use artificial intelligence
techniques and another 5.8 million developers expect to implement these
in near future \cite{evans}.

Machine learning is not just a business area in the United States,
survey results of 264 companies in the DACH region show, that 56 of them
already use that kind of technology in production. In the near future
112 companies plan to do so or already have initial experiences (see
figure \ref{img.crisp_ml_verbreitung}). It is seen by a fifth of the
decision-makers as a core area to improve the competitiveness and
profitability of companies in future. \cite{crisp}

\bilds{crisp_ml_verbreitung}{Distribution of machine learning of 264 companies in the DACH region \cite{crisp}}{Distribution of machine learning in 264 companies (DACH region) \cite{crisp}}

At the same time more and more companies shift their business logic from
a monolithic design to microservices. Each service is dedicated to a
single task that can be developed, deployed, replaced and scaled
independently. Test results have shown that not only this architecture
can help reduce infrastructure costs
\cite{villamizar2}\cite{villamizar}, but also reduces complexity of the
code base and enables applications to dynamically adjust computing
resources on demand \cite{villamizar}.

The presented project combines these techniques and demonstrates a
prototype that is open-source and is supported by common cloud
providers. Developers can integrate their own solutions into the
platform or exchange components ad libitum.

\section{Structure}\label{structure}

TODO

\section{Problem Statement}\label{problem-statement}

TODO

\begin{itemize}
\tightlist
\item
  many different platforms
\item
  complex field, hard to learn
\item
  complex setup for many environments (design, training, auswertung,
  storage service, \ldots{})
\end{itemize}

\chapter{State of the Art}\label{state-of-the-art}

\section{Machine Learning}\label{machine-learning}

\emph{Machine learning---the process by which computers can get better
at performing tasks through exposure to data, rather than through
explicit programming---requires massive computational power, the kind
usually found in clusters of energy-guzzling, cloud-based computer
servers outfitted with specialized processors. But an emerging trend
promises to bring the power of machine learning to mobile devices that
may lack or have only intermittent online connectivity. This will give
rise to machines that sense, perceive, learn from, and respond to their
environment and their users, enabling the emergence of new product
categories, reshaping how businesses engage with customers, and
transforming how work gets done across
industries.(https://www2.deloitte.com/insights/us/en/focus/signals-for-strategists/machine-learning-mobile-applications.html)}
TODO CITATION

\subsection{Classification}\label{classification}

\subsubsection{LATEX Symbol classification
application}\label{latex-symbol-classification-application}

\begin{figure}
\centering
\includegraphics{./images/latex-symbol-classifier.png}
\caption{\LaTeX symbol classification app}
\end{figure}

\subsection{Neural Network Frameworks}\label{neural-network-frameworks}

\subsubsection{Tensorflow}\label{tensorflow}

\subsubsection{Deeplearning4J}\label{deeplearning4j}

\section{Container Orchestration}\label{container-orchestration}

\subsection{Docker Containers}\label{docker-containers}

Containers enable software developers to deploy applications that are
portable and consistent across different environments and providers
\cite{baier-kub} by running isolated on top of the operating system's
kernel \cite{bashari}. As an organisation, Docker\footnote{https://docker.com}
has seen an increase of popularity very quickly, mainly because of its
advantages, which are speed, portability, scalability, rapid delivery,
and density \cite{bashari}.

Building a Docker container is fast, because they are small and and
overhead of a virtual machine. The container format itself is
standardized, which means that developers only have to ensure that their
application runs inside the container, which is then bundled into a
single unit. The unit can be deployed on any Linux system as well as on
various cloud environments and therefore easily be scaled. Not using a
full operating system makes containers use less resources than virtual
machines, which ensures higher workloads with greater density.
\cite{joy2015}

\subsection{Microservices}\label{microservices}

TODO As we break down an application into very specific domains, we need
a uniform way to communicate between all the various pieces and domains.
Web services have served this purpose for years, but the added isolation
and granular focus that containers bring have paved a way for what is
being named microservices. \cite{baier-kub}

\begin{quote}
TODO In short, the microservice architectural style is an approach to
developing a single application as a suite of small services, each
running in its own process and communicating with lightweight
mechanisms, often an HTTP resource API. These services are built around
business capabilities and independently deployable by fully automated
deployment machinery. There is a bare minimum of centralized management
of these services, which may be written in different programming
languages and use different data storage technologies. (deloitte)
\end{quote}

\bild{monolithic_vs_microservice}{15cm}{Monolithic Architecture vs. Microservice Architecture}{Monolithic Architecture vs. Microservice Architecture}

\subsection{Comparison of Container Orchestration
Technologies}\label{comparison-of-container-orchestration-technologies}

\subsection{Kubernetes}\label{kubernetes}

TODO Kubernetes is a system, developed by Google, for managing
containerized applications across a cluster of nodes. The controlling
services in a Kubernetes cluster are called the master components and
have a number of unique services which are used to manage a cluster's
workload and communications across the system\cite{kub_intro}.

\subsection{Docker Swarm}\label{docker-swarm}

\subsection{Kubernetes}\label{kubernetes-1}

\chapter{Requirements}\label{requirements}

\section{Functional Requirements}\label{functional-requirements}

\section{Non-Functional Requirements}\label{non-functional-requirements}

\chapter{Specification}\label{specification}

\section{Neural Network Objects}\label{neural-network-objects}

\subsection{State of Neural Network
Objects}\label{state-of-neural-network-objects}

\bild{nn-states}{15cm}{State Machine of a Neural Network}{State Machine of a Neural Network}

\chapter{Designing the API}\label{designing-the-api}

\section{REST API Documentation}\label{rest-api-documentation}

\subsection{Base URL}\label{base-url}

\begin{verbatim}
http[s]://<clusterip>
\end{verbatim}

\subsection{Import a new ViNNSL XML
Defintion}\label{import-a-new-vinnsl-xml-defintion}

\begin{verbatim}
POST /vinnsl
\end{verbatim}

\subsubsection{Parameters}\label{parameters}

\begin{longtable}[]{@{}llll@{}}
\toprule
Type & Name & Description & Schema\tabularnewline
\midrule
\endhead
\textbf{Body} & \textbf{vinnsl} \emph{required} & vinnsl &
Vinnsl\tabularnewline
\bottomrule
\end{longtable}

\subsubsection{Responses}\label{responses}

\begin{longtable}[]{@{}lll@{}}
\toprule
HTTP Code & Description & Schema\tabularnewline
\midrule
\endhead
\textbf{201} & Created & No Content\tabularnewline
\textbf{500} & Server Error & Error\tabularnewline
\bottomrule
\end{longtable}

\subsubsection{Consumes}\label{consumes}

\begin{itemize}
\tightlist
\item
  \texttt{application/xml}
\end{itemize}

\subsubsection{Produces}\label{produces}

\begin{itemize}
\tightlist
\item
  \texttt{*/*}
\end{itemize}

\subsubsection{Tags}\label{tags}

\begin{itemize}
\tightlist
\item
  vinnsl-service-controller
\end{itemize}

\subsubsection{Example HTTP request}\label{example-http-request}

\paragraph{Header}\label{header}

\begin{verbatim}
Content-Type: application/xml
\end{verbatim}

\paragraph{Body}\label{body}

\begin{verbatim}
<vinnsl>
  <description>
    <identifier><!-- will be generated --></identifier>
    <metadata>
      <paradigm>classification</paradigm>
      <name>Backpropagation Classification</name>
      <description>Iris Classification Example</description>
      <version>
        <major>1</major>
        <minor>0</minor>
      </version>
    </metadata>
    <creator>
      <name>Ronald Fisher</name>
      <contact>ronald.fisher@institution.com</contact>
    </creator>
    <problemDomain>
      <propagationType type="feedforward">
        <learningType>supervised</learningType>
      </propagationType>
      <applicationField>Classification</applicationField>
      <networkType>Backpropagation</networkType>
      <problemType>Classifiers</problemType>
    </problemDomain>
    <endpoints>
      <train>true</train>
      <retrain>true</retrain>
      <evaluate>true</evaluate>
    </endpoints>
    <structure>
       <input>
        <ID>Input1</ID>
        <size>
            <min>4</min>
            <max>4</max>
        </size>
       </input>
       <hidden>
        <ID>Hidden1</ID>
        <size>
            <min>3</min>
            <max>3</max>
        </size>
       </hidden>
       <hidden>
        <ID>Hidden2</ID>
        <size>
            <min>3</min>
            <max>3</max>
        </size>
       </hidden>
       <output>
        <ID>Output1</ID>
        <size>
            <min>3</min>
            <max>3</max>
        </size>
       </output>
     </structure>
     <parameters/>
     <data>
        <description>iris txt file with 3 classifications, 4 input vars</description>
        <tabledescription>no input as table possible</tabledescription>
        <filedescription>CSV file</filedescription>
     </data>
  </description>
</vinnsl>
\end{verbatim}

\subsubsection{Example HTTP response}\label{example-http-response}

Statuscode: \texttt{201} CREATED

\paragraph{Header}\label{header-1}

\begin{verbatim}
Location: https://<baseURL>/vinnsl/5ade36bbd601800001206798
\end{verbatim}

\subsection{List all Neural Networks}\label{list-all-neural-networks}

\begin{verbatim}
GET /vinnsl
\end{verbatim}

\subsubsection{Responses}\label{responses-1}

\begin{longtable}[]{@{}lll@{}}
\toprule
HTTP Code & Description & Schema\tabularnewline
\midrule
\endhead
\textbf{200} & OK & \textless{} Vinnsl \textgreater{}
array\tabularnewline
\textbf{404} & Not Found & No Content\tabularnewline
\textbf{500} & Server Error & Error\tabularnewline
\bottomrule
\end{longtable}

\subsubsection{Produces}\label{produces-1}

\begin{itemize}
\tightlist
\item
  \texttt{application/json}
\end{itemize}

\subsubsection{Tags}\label{tags-1}

\begin{itemize}
\tightlist
\item
  vinnsl-service-controller
\end{itemize}

\subsubsection{Example HTTP Response}\label{example-http-response-1}

\begin{verbatim}
[
    {
        "identifier": "5ab91658e8cc45946600ea11",
        "description": {},
        "definition": {},
        "data": {},
        "instance": {},
        "trainingresult": {},
        "result": {},
        "nncloud": {
            "status": "CREATED",
            "dl4jNetwork": "{}
        }
    },
    ...
]
\end{verbatim}

\subsection{Delete all Neural
Networks}\label{delete-all-neural-networks}

\begin{verbatim}
DELETE /vinnsl/deleteall
\end{verbatim}

\subsubsection{Responses}\label{responses-2}

\begin{longtable}[]{@{}lll@{}}
\toprule
HTTP Code & Description & Schema\tabularnewline
\midrule
\endhead
\textbf{200} & OK & object\tabularnewline
\textbf{204} & No Content & No Content\tabularnewline
\textbf{500} & Server Error & Error\tabularnewline
\bottomrule
\end{longtable}

\subsubsection{Produces}\label{produces-2}

\begin{itemize}
\tightlist
\item
  \texttt{application/json}
\end{itemize}

\subsubsection{Tags}\label{tags-2}

\begin{itemize}
\tightlist
\item
  vinnsl-service-controller
\end{itemize}

\subsection{Get Neural Network Object}\label{get-neural-network-object}

\begin{verbatim}
GET /vinnsl/{id}
\end{verbatim}

\subsubsection{Parameters}\label{parameters-1}

\begin{longtable}[]{@{}llll@{}}
\toprule
Type & Name & Description & Schema\tabularnewline
\midrule
\endhead
\textbf{Path} & \textbf{id} \emph{required} & id & string\tabularnewline
\bottomrule
\end{longtable}

\subsubsection{Responses}\label{responses-3}

\begin{longtable}[]{@{}lll@{}}
\toprule
HTTP Code & Description & Schema\tabularnewline
\midrule
\endhead
\textbf{200} & OK & Vinnsl\tabularnewline
\textbf{404} & Not Found & No Content\tabularnewline
\bottomrule
\end{longtable}

\subsubsection{Produces}\label{produces-3}

\begin{itemize}
\tightlist
\item
  \texttt{application/xml}
\item
  \texttt{application/json}
\end{itemize}

\subsubsection{Tags}\label{tags-3}

\begin{itemize}
\tightlist
\item
  vinnsl-service-controller
\end{itemize}

\subsubsection{Example HTTP response}\label{example-http-response-2}

\begin{verbatim}
<?xml version="1.0" encoding="UTF-8" standalone="yes"?>
<vinnsl>
    <identifier>5ab91658e8cc45946600ea11</identifier>
    <description>
        <identifier></identifier>
        <metadata>
            <paradigm>classification</paradigm>
            <name>Backpropagation Classification</name>
            <description>Face Recognition Example</description>
            <version>
                <major>1</major>
                <minor>5</minor>
            </version>
        </metadata>
        <creator>
            <name>Autor 1</name>
            <contact>author1@institution.com</contact>
        </creator>
        <problemDomain>
            <propagationType type="feedforward">
                <learningType>supervised</learningType>
            </propagationType>
            <applicationField>EMS</applicationField>
            <applicationField>Operations</applicationField>
            <applicationField>FaceRecoginition</applicationField>
            <networkType>Backpropagation</networkType>
            <problemType>Classifiers</problemType>
        </problemDomain>
        <endpoints>
            <train>true</train>
            <retrain>true</retrain>
            <evaluate>true</evaluate>
        </endpoints>
        <structure>
            <input>
                <ID>Input1</ID>
                <dimension>
                    <min>1</min>
                    <max>1</max>
                </dimension>
                <size>
                    <min>960</min>
                    <max>960</max>
                </size>
            </input>
            <hidden>
                <ID>Hidden1</ID>
                <dimension>
                    <min>1</min>
                    <max>1024</max>
                </dimension>
            </hidden>
            <output>
                <ID>Output1</ID>
                <dimension>
                    <min>1</min>
                    <max>1</max>
                </dimension>
                <size>
                    <min>1</min>
                    <max>1</max>
                </size>
            </output>
        </structure>
        <parameters/>
        <data>
            <description>Input are face images with 32x30 px</description>
            <tabledescription>no input as table possible</tabledescription>
            <filedescription>prepare the input as file by reading the image files</filedescription>
        </data>
    </description>
    <definition>
        <identifier></identifier>
        <problemDomain>
            <propagationType type="feedforward">
                <learningType>supervised</learningType>
            </propagationType>
            <applicationField>EMS</applicationField>
            <applicationField>Operations</applicationField>
            <applicationField>FaceRecoginition</applicationField>
            <networkType>Backpropagation</networkType>
            <problemType>Classifiers</problemType>
        </problemDomain>
        <endpoints></endpoints>
        <executionEnvironment>
            <serial>true</serial>
        </executionEnvironment>
        <structure>
            <input>
                <ID>Input1</ID>
                <dimension>1</dimension>
                <size>960</size>
            </input>
            <hidden>
                <ID>Hidden1</ID>
                <dimension>1</dimension>
                <size>1024</size>
            </hidden>
            <output>
                <ID>Output1</ID>
                <dimension>1</dimension>
                <size>1</size>
            </output>
            <connections/>
        </structure>
        <resultSchema>
            <instance>true</instance>
            <training>true</training>
        </resultSchema>
        <parameters>
            <valueparameter name="learningrate">0.4</valueparameter>
            <valueparameter name="biasInput">1</valueparameter>
            <valueparameter name="biasHidden">1</valueparameter>
            <valueparameter name="momentum">0.1</valueparameter>
            <comboparameter name="ativationfunction">sigmoid</comboparameter>
            <valueparameter name="threshold">0.00001</valueparameter>
            <comboparameter name="activationfunction">sigmoid</comboparameter>
        </parameters>
        <data>
            <description>Input are face images with 32x30 px</description>
            <dataSchemaID>iris.txt</dataSchemaID>
        </data>
    </definition>
    <data>
        <identifier>5ab4e69c8f136a16bf81f093</identifier>
        <data>
            <file>5ab4e69c8f136a16bf81f093</file>
        </data>
    </data>
</vinnsl>
\end{verbatim}

\subsection{Remove Neural Network
Object}\label{remove-neural-network-object}

\begin{verbatim}
DELETE /vinnsl/{id}
\end{verbatim}

\subsubsection{Parameters}\label{parameters-2}

\begin{longtable}[]{@{}llll@{}}
\toprule
Type & Name & Description & Schema\tabularnewline
\midrule
\endhead
\textbf{Path} & \textbf{id} \emph{required} & id & string\tabularnewline
\bottomrule
\end{longtable}

\subsubsection{Responses}\label{responses-4}

\begin{longtable}[]{@{}lll@{}}
\toprule
HTTP Code & Description & Schema\tabularnewline
\midrule
\endhead
\textbf{200} & OK & ResponseEntity\tabularnewline
\textbf{204} & No Content & No Content\tabularnewline
\textbf{500} & Server Error & No Content\tabularnewline
\bottomrule
\end{longtable}

\subsubsection{Produces}\label{produces-4}

\begin{itemize}
\tightlist
\item
  \texttt{*/*}
\end{itemize}

\subsubsection{Tags}\label{tags-4}

\begin{itemize}
\tightlist
\item
  vinnsl-service-controller
\end{itemize}

\subsection{Add/Replace File of Neural
Network}\label{addreplace-file-of-neural-network}

\begin{verbatim}
PUT /vinnsl/{id}/addfile
\end{verbatim}

\subsubsection{Parameters}\label{parameters-3}

\begin{longtable}[]{@{}llll@{}}
\toprule
Type & Name & Description & Schema\tabularnewline
\midrule
\endhead
\textbf{Path} & \textbf{id} \emph{required} & id & string\tabularnewline
\textbf{Query} & \textbf{fileId} \emph{required} & fileId &
string\tabularnewline
\bottomrule
\end{longtable}

\subsubsection{Responses}\label{responses-5}

\begin{longtable}[]{@{}lll@{}}
\toprule
HTTP Code & Description & Schema\tabularnewline
\midrule
\endhead
\textbf{200} & OK & Vinnsl\tabularnewline
\textbf{404} & Not Found & No Content\tabularnewline
\textbf{500} & Server Error & Error\tabularnewline
\bottomrule
\end{longtable}

\subsubsection{Consumes}\label{consumes-1}

\begin{itemize}
\tightlist
\item
  \texttt{application/json}
\end{itemize}

\subsubsection{Produces}\label{produces-5}

\begin{itemize}
\tightlist
\item
  \texttt{application/xml}
\item
  \texttt{application/json}
\end{itemize}

\subsubsection{Tags}\label{tags-5}

\begin{itemize}
\tightlist
\item
  vinnsl-service-controller
\end{itemize}

\subsection{Add/Replace ViNNSL Definition of Neural
Network}\label{addreplace-vinnsl-definition-of-neural-network}

\begin{verbatim}
PUT /vinnsl/{id}/definition
\end{verbatim}

\subsubsection{Parameters}\label{parameters-4}

\begin{longtable}[]{@{}llll@{}}
\toprule
Type & Name & Description & Schema\tabularnewline
\midrule
\endhead
\textbf{Path} & \textbf{id} \emph{required} & id & string\tabularnewline
\textbf{Body} & \textbf{def} \emph{required} & def &
Definition\tabularnewline
\bottomrule
\end{longtable}

\subsubsection{Responses}\label{responses-6}

\begin{longtable}[]{@{}lll@{}}
\toprule
HTTP Code & Description & Schema\tabularnewline
\midrule
\endhead
\textbf{200} & OK & Vinnsl\tabularnewline
\textbf{404} & Not Found & No Content\tabularnewline
\textbf{500} & Server Error & Error\tabularnewline
\bottomrule
\end{longtable}

\subsubsection{Consumes}\label{consumes-2}

\begin{itemize}
\tightlist
\item
  \texttt{application/xml}
\item
  \texttt{application/json}
\end{itemize}

\subsubsection{Produces}\label{produces-6}

\begin{itemize}
\tightlist
\item
  \texttt{*/*}
\end{itemize}

\subsubsection{Tags}\label{tags-6}

\begin{itemize}
\tightlist
\item
  vinnsl-service-controller
\end{itemize}

\subsubsection{Example HTTP request}\label{example-http-request-1}

\paragraph{Request body}\label{request-body}

\begin{verbatim}
<definition>
<identifier><!-- will be generated --></identifier>
<metadata>
  <paradigm>classification</paradigm>
  <name>Backpropagation Classification</name>
  <description>Iris Classification Example</description>
  <version>
    <major>1</major>
    <minor>0</minor>
  </version>
</metadata>
<creator>
  <name>Ronald Fisher</name>
  <contact>ronald.fisher@institution.com</contact>
</creator>
<problemDomain>
  <propagationType type="feedforward">
    <learningType>supervised</learningType>
  </propagationType>
  <applicationField>Classification</applicationField>
  <networkType>Backpropagation</networkType>
  <problemType>Classifiers</problemType>
</problemDomain>
<endpoints>
  <train>true</train>
</endpoints>
<executionEnvironment>
    <serial>true</serial>
</executionEnvironment>
<structure>
   <input>
    <ID>Input1</ID>
    <size>4</size>
   </input>
   <hidden>
    <ID>Hidden1</ID>
    <size>3</size>
   </hidden>
   <hidden>
    <ID>Hidden2</ID>
    <size>3</size>
   </hidden>
   <output>
    <ID>Output1</ID>
    <size>3</size>
   </output>
   <connections>
    <!--<fullconnected>
        <fromblock>Input1</fromblock>
        <toblock>Hidden1</toblock>
        <fromblock>Hidden1</fromblock>
        <toblock>Output1</toblock>
    </fullconnected>-->
   </connections>
 </structure>
 <resultSchema>
    <instance>true</instance>
    <training>true</training>
 </resultSchema>
 <parameters>
    <valueparameter name="learningrate">0.1</valueparameter>
    <comboparameter name="activationfunction">tanh</comboparameter>
    <valueparameter name="iterations">500</valueparameter>
    <valueparameter name="seed">6</valueparameter>
 </parameters>
 <data>
    <description>iris txt file with 3 classifications, 4 input vars</description>
    <dataSchemaID>name/iris.txt</dataSchemaID>
 </data>
</definition>
\end{verbatim}

\subsection{Add/Replace ViNNSL Instanceschema of Neural
Network}\label{addreplace-vinnsl-instanceschema-of-neural-network}

\begin{verbatim}
PUT /vinnsl/{id}/instanceschema
\end{verbatim}

\subsubsection{Parameters}\label{parameters-5}

\begin{longtable}[]{@{}llll@{}}
\toprule
Type & Name & Description & Schema\tabularnewline
\midrule
\endhead
\textbf{Path} & \textbf{id} \emph{required} & id & string\tabularnewline
\textbf{Body} & \textbf{instance} \emph{required} & instance &
Instanceschema\tabularnewline
\bottomrule
\end{longtable}

\subsubsection{Responses}\label{responses-7}

\begin{longtable}[]{@{}lll@{}}
\toprule
HTTP Code & Description & Schema\tabularnewline
\midrule
\endhead
\textbf{200} & OK & object\tabularnewline
\textbf{404} & Not Found & No Content\tabularnewline
\textbf{500} & Server Error & Error\tabularnewline
\bottomrule
\end{longtable}

\subsubsection{Consumes}\label{consumes-3}

\begin{itemize}
\tightlist
\item
  \texttt{application/xml}
\item
  \texttt{application/json}
\end{itemize}

\subsubsection{Produces}\label{produces-7}

\begin{itemize}
\tightlist
\item
  \texttt{*/*}
\end{itemize}

\subsubsection{Tags}\label{tags-7}

\begin{itemize}
\tightlist
\item
  vinnsl-service-controller
\end{itemize}

\subsubsection{Example HTTP request}\label{example-http-request-2}

\paragraph{Request body}\label{request-body-1}

\begin{verbatim}
<instanceschema>
</instanceschema>
\end{verbatim}

\subsection{Add/Replace ViNNSL Resultschema of Neural
Network}\label{addreplace-vinnsl-resultschema-of-neural-network}

\begin{verbatim}
PUT /vinnsl/{id}/resultschema
\end{verbatim}

\subsubsection{Parameters}\label{parameters-6}

\begin{longtable}[]{@{}llll@{}}
\toprule
Type & Name & Description & Schema\tabularnewline
\midrule
\endhead
\textbf{Path} & \textbf{id} \emph{required} & id & string\tabularnewline
\textbf{Body} & \textbf{resultSchema} \emph{required} & resultSchema &
Resultschema\tabularnewline
\bottomrule
\end{longtable}

\subsubsection{Responses}\label{responses-8}

\begin{longtable}[]{@{}lll@{}}
\toprule
HTTP Code & Description & Schema\tabularnewline
\midrule
\endhead
\textbf{200} & OK & object\tabularnewline
\textbf{404} & Not Found & No Content\tabularnewline
\textbf{500} & Server Error & Error\tabularnewline
\bottomrule
\end{longtable}

\subsubsection{Consumes}\label{consumes-4}

\begin{itemize}
\tightlist
\item
  \texttt{application/xml}
\item
  \texttt{application/json}
\end{itemize}

\subsubsection{Produces}\label{produces-8}

\begin{itemize}
\tightlist
\item
  \texttt{*/*}
\end{itemize}

\subsubsection{Tags}\label{tags-8}

\begin{itemize}
\tightlist
\item
  vinnsl-service-controller
\end{itemize}

\subsubsection{Example HTTP request}\label{example-http-request-3}

\paragraph{Request body}\label{request-body-2}

\begin{verbatim}
<resultschema>
</resultschema>
\end{verbatim}

\subsection{Add/Replace ViNNSL Trainingresult of Neural
Network}\label{addreplace-vinnsl-trainingresult-of-neural-network}

\begin{verbatim}
PUT /vinnsl/{id}/trainingresult
\end{verbatim}

\subsubsection{Parameters}\label{parameters-7}

\begin{longtable}[]{@{}llll@{}}
\toprule
Type & Name & Description & Schema\tabularnewline
\midrule
\endhead
\textbf{Path} & \textbf{id} \emph{required} & id & string\tabularnewline
\textbf{Body} & \textbf{trainingresult} \emph{required} & trainingresult
& Trainingresultschema\tabularnewline
\bottomrule
\end{longtable}

\subsubsection{Responses}\label{responses-9}

\begin{longtable}[]{@{}lll@{}}
\toprule
HTTP Code & Description & Schema\tabularnewline
\midrule
\endhead
\textbf{200} & OK & object\tabularnewline
\textbf{404} & Not Found & No Content\tabularnewline
\textbf{500} & Server Error & Error\tabularnewline
\bottomrule
\end{longtable}

\subsubsection{Consumes}\label{consumes-5}

\begin{itemize}
\tightlist
\item
  \texttt{application/xml}
\item
  \texttt{application/json}
\end{itemize}

\subsubsection{Produces}\label{produces-9}

\begin{itemize}
\tightlist
\item
  \texttt{*/*}
\end{itemize}

\subsubsection{Tags}\label{tags-9}

\begin{itemize}
\tightlist
\item
  vinnsl-service-controller
\end{itemize}

\subsubsection{Example HTTP request}\label{example-http-request-4}

\paragraph{Request body}\label{request-body-3}

\begin{verbatim}
<trainingresult>
</trainingresult>
\end{verbatim}

\subsection{Get Status of all Neural
Networks}\label{get-status-of-all-neural-networks}

\begin{verbatim}
GET /status
\end{verbatim}

\subsubsection{Responses}\label{responses-10}

\begin{longtable}[]{@{}lll@{}}
\toprule
HTTP Code & Description & Schema\tabularnewline
\midrule
\endhead
\textbf{200} & OK & object\tabularnewline
\textbf{404} & Not Found & No Content\tabularnewline
\bottomrule
\end{longtable}

\subsubsection{Produces}\label{produces-10}

\begin{itemize}
\tightlist
\item
  \texttt{application/json}
\end{itemize}

\subsubsection{Tags}\label{tags-10}

\begin{itemize}
\tightlist
\item
  nn-status-controller
\end{itemize}

\subsubsection{HTTP response example}\label{http-response-example}

\begin{verbatim}
{
    "5ab91658e8cc45946600ea11": "INPROGRESS"
}
\end{verbatim}

\subsection{Get Status of Neural
Network}\label{get-status-of-neural-network}

\begin{verbatim}
GET /status/{id}
\end{verbatim}

\subsubsection{Parameters}\label{parameters-8}

\begin{longtable}[]{@{}llll@{}}
\toprule
Type & Name & Description & Schema\tabularnewline
\midrule
\endhead
\textbf{Path} & \textbf{id} \emph{required} & id & string\tabularnewline
\bottomrule
\end{longtable}

\subsubsection{Responses}\label{responses-11}

\begin{longtable}[]{@{}lll@{}}
\toprule
HTTP Code & Description & Schema\tabularnewline
\midrule
\endhead
\textbf{200} & OK & object\tabularnewline
\textbf{404} & Not Found & No Content\tabularnewline
\bottomrule
\end{longtable}

\subsubsection{Produces}\label{produces-11}

\begin{itemize}
\tightlist
\item
  \texttt{application/json}
\end{itemize}

\subsubsection{Tags}\label{tags-11}

\begin{itemize}
\tightlist
\item
  nn-status-controller
\end{itemize}

\subsection{Set Status of a Neural
Network}\label{set-status-of-a-neural-network}

\begin{verbatim}
PUT /status/{id}/{status}
\end{verbatim}

\subsubsection{Parameters}\label{parameters-9}

\begin{longtable}[]{@{}llll@{}}
\toprule
\begin{minipage}[b]{0.08\columnwidth}\raggedright\strut
Type\strut
\end{minipage} & \begin{minipage}[b]{0.21\columnwidth}\raggedright\strut
Name\strut
\end{minipage} & \begin{minipage}[b]{0.11\columnwidth}\raggedright\strut
Description\strut
\end{minipage} & \begin{minipage}[b]{0.49\columnwidth}\raggedright\strut
Schema\strut
\end{minipage}\tabularnewline
\midrule
\endhead
\begin{minipage}[t]{0.08\columnwidth}\raggedright\strut
\textbf{Path}\strut
\end{minipage} & \begin{minipage}[t]{0.21\columnwidth}\raggedright\strut
\textbf{id} \emph{required}\strut
\end{minipage} & \begin{minipage}[t]{0.11\columnwidth}\raggedright\strut
id\strut
\end{minipage} & \begin{minipage}[t]{0.49\columnwidth}\raggedright\strut
string\strut
\end{minipage}\tabularnewline
\begin{minipage}[t]{0.08\columnwidth}\raggedright\strut
\textbf{Path}\strut
\end{minipage} & \begin{minipage}[t]{0.21\columnwidth}\raggedright\strut
\textbf{status} \emph{required}\strut
\end{minipage} & \begin{minipage}[t]{0.11\columnwidth}\raggedright\strut
status\strut
\end{minipage} & \begin{minipage}[t]{0.49\columnwidth}\raggedright\strut
enum (\texttt{CREATED,\ QUEUED,\ INPROGRESS,\ FINISHED,\ ERROR})\strut
\end{minipage}\tabularnewline
\bottomrule
\end{longtable}

\subsubsection{Responses}\label{responses-12}

\begin{longtable}[]{@{}lll@{}}
\toprule
HTTP Code & Description & Schema\tabularnewline
\midrule
\endhead
\textbf{200} & OK & object\tabularnewline
\textbf{404} & Not Found & No Content\tabularnewline
\textbf{500} & Server Error & Error\tabularnewline
\bottomrule
\end{longtable}

\subsubsection{Consumes}\label{consumes-6}

\begin{itemize}
\tightlist
\item
  \texttt{application/json}
\end{itemize}

\subsubsection{Produces}\label{produces-12}

\begin{itemize}
\tightlist
\item
  \texttt{application/json}
\end{itemize}

\subsubsection{Tags}\label{tags-12}

\begin{itemize}
\tightlist
\item
  nn-status-controller
\end{itemize}

\subsection{Get Deeplearning4J Transformation Object of Neural
Network}\label{get-deeplearning4j-transformation-object-of-neural-network}

\begin{verbatim}
GET /dl4j/{id}
\end{verbatim}

\subsubsection{Parameters}\label{parameters-10}

\begin{longtable}[]{@{}llll@{}}
\toprule
Type & Name & Description & Schema\tabularnewline
\midrule
\endhead
\textbf{Path} & \textbf{id} \emph{required} & id & string\tabularnewline
\bottomrule
\end{longtable}

\subsubsection{Responses}\label{responses-13}

\begin{longtable}[]{@{}lll@{}}
\toprule
HTTP Code & Description & Schema\tabularnewline
\midrule
\endhead
\textbf{200} & OK & string\tabularnewline
\textbf{404} & Not Found & No Content\tabularnewline
\bottomrule
\end{longtable}

\subsubsection{Produces}\label{produces-13}

\begin{itemize}
\tightlist
\item
  \texttt{application/json}
\end{itemize}

\subsubsection{Tags}\label{tags-13}

\begin{itemize}
\tightlist
\item
  dl4j-service-controller
\end{itemize}

\subsection{Put Deeplearning4J Transformation Object of Neural
Network}\label{put-deeplearning4j-transformation-object-of-neural-network}

\begin{verbatim}
PUT /dl4j/{id}
\end{verbatim}

\subsubsection{Parameters}\label{parameters-11}

\begin{longtable}[]{@{}llll@{}}
\toprule
Type & Name & Description & Schema\tabularnewline
\midrule
\endhead
\textbf{Path} & \textbf{id} \emph{required} & id & string\tabularnewline
\textbf{Body} & \textbf{dl4J} \emph{required} & dl4J &
string\tabularnewline
\bottomrule
\end{longtable}

\subsubsection{Responses}\label{responses-14}

\begin{longtable}[]{@{}lll@{}}
\toprule
HTTP Code & Description & Schema\tabularnewline
\midrule
\endhead
\textbf{200} & OK & ResponseEntity\tabularnewline
\textbf{404} & Not Found & No Content\tabularnewline
\textbf{500} & Server Error & Error\tabularnewline
\bottomrule
\end{longtable}

\subsubsection{Consumes}\label{consumes-7}

\begin{itemize}
\tightlist
\item
  \texttt{application/json}
\end{itemize}

\subsubsection{Produces}\label{produces-14}

\begin{itemize}
\tightlist
\item
  \texttt{application/json}
\end{itemize}

\subsubsection{Tags}\label{tags-14}

\begin{itemize}
\tightlist
\item
  dl-4j-service-controller
\end{itemize}

\chapter{Implementation of a
Prototype}\label{implementation-of-a-prototype}

\section{User Interface}\label{user-interface}

\bild{vinnsl-nn-ui}{15cm}{User Interface of Prototype}{User Interface of Prototype}

\chapter{Use Cases}\label{use-cases}

\begin{enumerate}
\def\labelenumi{\arabic{enumi})}
\item
  iris classification
\item
  MNIST?
\item
  hosted trained network
\end{enumerate}

\chapter{Future Work}\label{future-work}

TODO

\begin{itemize}
\tightlist
\item
  more function\\
\item
  backend für tensorflow
\item
  grafischer NN designer
\item
  trainierte netzwerke als webservice veröffentlichen
\end{itemize}

\chapter{Conclusions}\label{conclusions}

\chapter{Acknowledgments}\label{acknowledgments}

\chapter{Dedication}\label{dedication}

\chapter{Appendices}\label{appendices}
